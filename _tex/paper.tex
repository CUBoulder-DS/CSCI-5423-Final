% Options for packages loaded elsewhere
\PassOptionsToPackage{unicode}{hyperref}
\PassOptionsToPackage{hyphens}{url}
\PassOptionsToPackage{dvipsnames,svgnames,x11names}{xcolor}
%
\documentclass[
  letterpaper,
  DIV=11,
  numbers=noendperiod]{scrreprt}

\usepackage{amsmath,amssymb}
\usepackage{iftex}
\ifPDFTeX
  \usepackage[T1]{fontenc}
  \usepackage[utf8]{inputenc}
  \usepackage{textcomp} % provide euro and other symbols
\else % if luatex or xetex
  \usepackage{unicode-math}
  \defaultfontfeatures{Scale=MatchLowercase}
  \defaultfontfeatures[\rmfamily]{Ligatures=TeX,Scale=1}
\fi
\usepackage{lmodern}
\ifPDFTeX\else  
    % xetex/luatex font selection
\fi
% Use upquote if available, for straight quotes in verbatim environments
\IfFileExists{upquote.sty}{\usepackage{upquote}}{}
\IfFileExists{microtype.sty}{% use microtype if available
  \usepackage[]{microtype}
  \UseMicrotypeSet[protrusion]{basicmath} % disable protrusion for tt fonts
}{}
\makeatletter
\@ifundefined{KOMAClassName}{% if non-KOMA class
  \IfFileExists{parskip.sty}{%
    \usepackage{parskip}
  }{% else
    \setlength{\parindent}{0pt}
    \setlength{\parskip}{6pt plus 2pt minus 1pt}}
}{% if KOMA class
  \KOMAoptions{parskip=half}}
\makeatother
\usepackage{xcolor}
\usepackage[margin=0.5in]{geometry}
\setlength{\emergencystretch}{3em} % prevent overfull lines
\setcounter{secnumdepth}{-\maxdimen} % remove section numbering
% Make \paragraph and \subparagraph free-standing
\makeatletter
\ifx\paragraph\undefined\else
  \let\oldparagraph\paragraph
  \renewcommand{\paragraph}{
    \@ifstar
      \xxxParagraphStar
      \xxxParagraphNoStar
  }
  \newcommand{\xxxParagraphStar}[1]{\oldparagraph*{#1}\mbox{}}
  \newcommand{\xxxParagraphNoStar}[1]{\oldparagraph{#1}\mbox{}}
\fi
\ifx\subparagraph\undefined\else
  \let\oldsubparagraph\subparagraph
  \renewcommand{\subparagraph}{
    \@ifstar
      \xxxSubParagraphStar
      \xxxSubParagraphNoStar
  }
  \newcommand{\xxxSubParagraphStar}[1]{\oldsubparagraph*{#1}\mbox{}}
  \newcommand{\xxxSubParagraphNoStar}[1]{\oldsubparagraph{#1}\mbox{}}
\fi
\makeatother


\providecommand{\tightlist}{%
  \setlength{\itemsep}{0pt}\setlength{\parskip}{0pt}}\usepackage{longtable,booktabs,array}
\usepackage{calc} % for calculating minipage widths
% Correct order of tables after \paragraph or \subparagraph
\usepackage{etoolbox}
\makeatletter
\patchcmd\longtable{\par}{\if@noskipsec\mbox{}\fi\par}{}{}
\makeatother
% Allow footnotes in longtable head/foot
\IfFileExists{footnotehyper.sty}{\usepackage{footnotehyper}}{\usepackage{footnote}}
\makesavenoteenv{longtable}
\usepackage{graphicx}
\makeatletter
\newsavebox\pandoc@box
\newcommand*\pandocbounded[1]{% scales image to fit in text height/width
  \sbox\pandoc@box{#1}%
  \Gscale@div\@tempa{\textheight}{\dimexpr\ht\pandoc@box+\dp\pandoc@box\relax}%
  \Gscale@div\@tempb{\linewidth}{\wd\pandoc@box}%
  \ifdim\@tempb\p@<\@tempa\p@\let\@tempa\@tempb\fi% select the smaller of both
  \ifdim\@tempa\p@<\p@\scalebox{\@tempa}{\usebox\pandoc@box}%
  \else\usebox{\pandoc@box}%
  \fi%
}
% Set default figure placement to htbp
\def\fps@figure{htbp}
\makeatother

\KOMAoption{captions}{tableheading}
\makeatletter
\@ifpackageloaded{caption}{}{\usepackage{caption}}
\AtBeginDocument{%
\ifdefined\contentsname
  \renewcommand*\contentsname{Table of contents}
\else
  \newcommand\contentsname{Table of contents}
\fi
\ifdefined\listfigurename
  \renewcommand*\listfigurename{List of Figures}
\else
  \newcommand\listfigurename{List of Figures}
\fi
\ifdefined\listtablename
  \renewcommand*\listtablename{List of Tables}
\else
  \newcommand\listtablename{List of Tables}
\fi
\ifdefined\figurename
  \renewcommand*\figurename{Figure}
\else
  \newcommand\figurename{Figure}
\fi
\ifdefined\tablename
  \renewcommand*\tablename{Table}
\else
  \newcommand\tablename{Table}
\fi
}
\@ifpackageloaded{float}{}{\usepackage{float}}
\floatstyle{ruled}
\@ifundefined{c@chapter}{\newfloat{codelisting}{h}{lop}}{\newfloat{codelisting}{h}{lop}[chapter]}
\floatname{codelisting}{Listing}
\newcommand*\listoflistings{\listof{codelisting}{List of Listings}}
\makeatother
\makeatletter
\makeatother
\makeatletter
\@ifpackageloaded{caption}{}{\usepackage{caption}}
\@ifpackageloaded{subcaption}{}{\usepackage{subcaption}}
\makeatother

\usepackage{bookmark}

\IfFileExists{xurl.sty}{\usepackage{xurl}}{} % add URL line breaks if available
\urlstyle{same} % disable monospaced font for URLs
\hypersetup{
  pdftitle={Modeling Persuasion Behaviors in Social Deduction Games},
  pdfauthor={First Author; Second Author},
  pdfkeywords={social deduction games, persuasion modeling, Werewolf
Among Us dataset, large language models, multimodal analysis},
  colorlinks=true,
  linkcolor={blue},
  filecolor={Maroon},
  citecolor={Blue},
  urlcolor={Blue},
  pdfcreator={LaTeX via pandoc}}


\title{Modeling Persuasion Behaviors in Social Deduction Games}
\author{First Author \and Second Author}
\date{2025-04-22}

\begin{document}
\maketitle
\begin{abstract}
This final project for CSCI‑5423 explores the modeling of persuasion
behaviors in the text-based social deduction game Werewolf. We leverage
the multimodal ``Werewolf Among Us'' dataset (Lai, 2022) and evaluate
several large language models on their ability to persuade, deceive, and
cooperate within game dialogues. Through a combination of feature
engineering, sequence modeling, and reinforcement learning agents, we
compare performance against baseline classifiers and analyze key
linguistic strategies.
\end{abstract}

\renewcommand*\contentsname{Table of contents}
{
\hypersetup{linkcolor=}
\setcounter{tocdepth}{2}
\tableofcontents
}

\chapter{Introduction}\label{introduction}

Introduce the motivation for studying persuasion in social deduction
games. Describe the Werewolf game mechanics and why it offers a rich
testbed for multimodal persuasion analysis.

\chapter{Related Work}\label{related-work}

Review prior datasets and frameworks:

\begin{itemize}
\tightlist
\item
  Werewolf Among Us: dataset and original paper (Lai, 2022)
\item
  PersuasionGames repository and baseline models
\item
  AmongAgents and evaluation of LLMs in interactive text-based games
  (Chi, 2024)
\item
  RL approaches to strategic play in Werewolf (Xu et al., 2023)
\item
  Werewolf Arena framework by Google for LLM evaluation
\end{itemize}

\chapter{Data and Preprocessing}\label{data-and-preprocessing}

Detail how we load and preprocess the HuggingFace Werewolf-Among-Us
dataset. Include data splits, feature extraction, and any augmentation
steps.

\chapter{Methods}\label{methods}

Describe the modeling approaches:

\begin{itemize}
\tightlist
\item
  Baseline classifiers (e.g., logistic regression, SVM)
\item
  Sequence models (e.g., LSTM, Transformer)
\item
  RL-based agent setup and reward definitions
\end{itemize}

\chapter{Experiments and Evaluation}\label{experiments-and-evaluation}

Outline experimental setups:

\begin{itemize}
\tightlist
\item
  Persuasion classification tasks
\item
  Game simulation with LLM agents
\item
  Metrics for persuasion success, deception detection, and cooperative
  play
\end{itemize}

\chapter{Results}\label{results}

Present quantitative results in tables and figures. Analyze which models
and features best capture persuasion strategies.

\chapter{Discussion}\label{discussion}

Interpret findings, discuss limitations, and propose future work.

\chapter{Conclusion}\label{conclusion}

Summarize contributions and insights from the project.

\chapter{References}\label{references}

Lai, B. (2022). Werewolf Among Us: A Multimodal Dataset for Modeling
Persuasion Behaviors in Social Deduction Games. arXiv:2212.08279.

Chi, Y. (2024). AmongAgents: Evaluating Large Language Models in the
Interactive Text-Based Social Deduction Game. arXiv:2407.16521v2.

Xu, Z., et al.~(2023). Language Agents with Reinforcement Learning for
Strategic Play in the Werewolf Game. arXiv:2310.18940v3.

Blanchard, T., et al.~(2024). Werewolf Arena: A Case Study in LLM
Evaluation via Social Deduction. arXiv:2407.13943.

\begin{center}\rule{0.5\linewidth}{0.5pt}\end{center}




\end{document}
